\documentclass[12pt, letterpaper]{article}
\usepackage[utf8]{inputenc}

% Allows equations and maths to be written out in a fluid way
\usepackage{amsmath}

% Remove indentations from the start of paragraphs
\usepackage[parfill]{parskip}

% Used for references, requires bibtex to be installed
\usepackage[backend=bibtex]{biblatex}
\addbibresource{intro.bib}

\usepackage{graphicx}
\graphicspath{ {images/} }
\DeclareGraphicsExtensions{.pdf,.jpeg,.png,.jpg}

\title{An introduction to \LaTeX}
\author{Samuel Watkins MEng \thanks{Gathered from information on the internet.}}
\date{March 2016}

\begin{document}

\begin{titlepage}
\pagenumbering{gobble}
\maketitle
\end{titlepage}

\pagenumbering{gobble}
\tableofcontents
\newpage

\pagenumbering{arabic}

\section{Introduction}
This is an introduction document. It doesn't contain anything interesting, just
the basic preamble.\cite{preamble}

In this example I'll be showing some simple things you can do in \LaTeX. The
following is some simple document formatting that allows you to structure your
paragraphs into sections, subsections, paragraphs, and subparagraphs.

Text formatting such as \textit{italics}, \textbf{bold}, and \underline{underlines} will not be provided with their own sections because it's not really necessary.\cite{formatting}

\section{Sections}
This is a section, they are all automatically numbered and titled.\cite{paragraphs}

\subsection{Subsections}
This is a subsection, their number is generated from the section that contains them.

\paragraph{}
This is an example of an untitled paragraph.

\paragraph{Paragraphs}
This is a paragraph, they are not numbered and their titles are put on the left of the text.

\subparagraph{Subparagraphs}
This is a subparagraph, they are indented from
their containing paragraph but are otherwise formatted in mostly the same way.

\newpage
\section{Lists}
Lists\cite{lists} can be ordered (numerically, alphbetically etc.) or unordered (bulleted lists). Below is an example of an unordered list:

\begin{itemize}
\item Unordered item \#1
\item Unordered item \#2
\begin{itemize}
\item Unordered sub-item \#1
\begin{enumerate}
\item Ordered sub-item \#1
\item Ordered sub-item \#2
\item Ordered sub-item \#3
\end{enumerate}
\item Unordered sub-item \#2
\end{itemize}
\item Unordered item \#3
\item Unordered item \#4
\item Unordered item \#5
\end{itemize}

And here is an example of an ordered list:
\begin{enumerate}
\item List item \#1
\begin{enumerate}
\item List sub-item \#1
\begin{itemize}
\item Unordered sub-item \#1
\item Unordered sub-item \#2
\end{itemize}
\item List item \#2
\end{enumerate}
\item List item \#2
\item List item \#3
\end{enumerate}

Note that not ony can you create easy nested lists, but you can also nest unordered lists in ordered lists and vice versa.

\newpage
\section{Equations}
It is possible to nicely format equations\cite{mathematics} in a fairly straightforward way like so:

\begin{equation*}
1 + 2 = 3
\end{equation*}

\begin{equation*}
3 - 2 = 1
\end{equation*}

\begin{equation*}
5 * 2 = 10
\end{equation*}

\begin{equation*}
f(x) = mx + c
\end{equation*}

\begin{equation*}
g(x) = \frac{1}{\sqrt{x^3}}x^2
\end{equation*}

\begin{equation*}
h(x) = \int\limits_0^n \frac{x_{n}}{\sqrt{x^3}}x^2 \ dx
\end{equation*}

All of the above equations were placed inside the equation environment.
They can be aligned by using the align environment like so:

\begin{align*}
1 + 2 &= 3\\
1 &= 3 - 2\\
g(x) &= \frac{1}{\sqrt{x^3}}x^2\\
\int\limits_0^n \frac{x_{n}}{\sqrt{x^3}}x^2 \ dx &= h(x) 
\end{align*}

Here I have made the alignment at the = sign, however I can make it align to any individual character.

\newpage
\section{Images}
It is possible to insert an image\cite{images} into the pdf document.
\begin{center}
\includegraphics{sampleSmall}
\end{center}

You can also resize, rotate, and position images conveniently:
Standard image:
\begin{center}
\includegraphics{sampleLarge}
\end{center}

Scaled down image:
\begin{center}
\includegraphics[scale=0.5]{sampleLarge}
\end{center}

\newpage
Resized image:
\begin{center}
\includegraphics[width=3cm, height=3cm]{sampleLarge}
\end{center}

Resscaled image relative to page size:
\begin{center}
\includegraphics[width=\textwidth]{sampleLarge}
\end{center}

\newpage
Rotated image:
\begin{center}
\includegraphics[angle=45, scale=0.5]{sampleLarge}
\end{center}

You can also add images to separate pages and reference them from the document. See image \ref{figure:1}
\begin{figure}[p!]
\centering
\includegraphics[scale=0.5]{sampleLarge}
\caption{Image referenced elsewhere in the document.}
\label{figure:1}
\end{figure}


\newpage
\section{Tables}
Here we will look at making tables\cite{tables}.

This table is centered and borderless:
\begin{center}
\begin{tabular}{ c c c }
    cell11 & cell12 & cell13 \\
    cell21 & cell22 & cell23 \\
    cell31 & cell32 & cell33
\end{tabular}
\end{center}

This table has a border:
\begin{center}
\begin{tabular}{ |c|c|c| }
    \hline
    cell11 & cell12 & cell13 \\
    \hline
    cell21 & cell22 & cell23 \\
    cell31 & cell32 & cell33 \\
    \hline
\end{tabular}
\end{center}

The formatting method means you can do some interesting things such as this:
\begin{center}
\begin{tabular}{ ||c c c||}
    \hline
    cell11 & cell12 & cell13 \\
    \hline\hline
    cell21 & cell22 & cell23 \\
    cell31 & cell32 & cell33 \\
    \hline
\end{tabular}
\end{center}

You can also place pages on a separate table page, see table \ref{table:1}
\begin{table}[p!]
\centering
\begin{tabular}{ ||c|c|c||}
    \hline
    Title1 & Title2 & Title3 \\
    \hline\hline
    item & item & item \\
    item & item & item \\
    \hline
\end{tabular}
\caption{Table referenced elsewhere in the document.}
\label{table:1}
\end{table}

\newpage
\printbibliography

\end{document}
